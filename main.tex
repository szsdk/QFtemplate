% !Mode:: "TeX:UTF-8"
\documentclass{qftemplate}

\usepackage{siunitx}
\usepackage{metalogo}

\author{萃英学院\ 沈周}
\title{qftemplate模板使用}

\begin{document}
\maketitle
\section{简介}
这是兰州大学2015年秋季学期的\LaTeX 模板。
\section{模板使用}
\begin{question}
如何使用此模板?
\end{question}
\begin{proof}[解]
    将{\tt qftemplate.cls}文件放在作业{\tt .tex}文档的目录下,在作业文档第一行加入
    \begin{verbatim}
\documentclass{qftemplate}
    \end{verbatim}
    就可以调用此模板,{\it 并用 \XeLaTeX 编译文件}。
\end{proof}

\begin{question}[输入问题与解答]
    如何正确的输入问题与解答?
\end{question}
\begin{proof}[解]
    此模板中已经定义了输入问题的环境{\tt question},可以按如下方式在正文中使用:\marginnote{其中中括号内的内容可以省略。}
\begin{verbatim}
\begin{question}[输入问题与解答]
如何正确的输入问题与解答?
\end{question}
\end{verbatim}
以及一个解答环境{\tt proof},使用方式与问题环境的使用类似:\marginnote{若省略这个中括号中的内容,证明的题头默认是{\it Proof}。}
\begin{verbatim}
\begin{proof}[证明]
    略。
\end{proof}
\end{verbatim}
\end{proof}

\begin{question}[边注释]
    如何在此模板中使用边注释?
\end{question}
\begin{proof}[解]
    在需要边注释处用以下的方式插入:
\begin{verbatim}
\marignnote{这是边注释}
\end{verbatim}
但为了得到正确的显示结果,需要将文档编译两遍。
\end{proof}

\begin{question}[关闭边注释]
    如何在此模板中关闭边注释?
\end{question}
\begin{proof}[解]
    在第一行声明文档类型时加入参数{\tt [nomargin]},即把第一行改为
\begin{verbatim}
\documentclass[nomargin]{qftemplate}
\end{verbatim}
\end{proof}

\begin{question}[输入单位]
如何正确的输入物理量:\SI{12}{GeV}?
\end{question}
\begin{proof}[解]
    在导言区输入
\begin{verbatim}
\usepackage{siunitx}
\end{verbatim}
然后在正文中使用
\begin{verbatim}
\SI{12}{GeV}
\end{verbatim}
\end{proof}

\end{document}
